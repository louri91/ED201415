\input{preambuloSimple.tex}

%----------------------------------------------------------------------------------------
%	TÍTULO Y DATOS DEL ALUMNO
%----------------------------------------------------------------------------------------

\title{	
\normalfont \normalsize 
\textsc{{\bf Estructuras de Datos (2014-2015)} \\ Grado en Ingeniería Informática \\ Universidad de Granada} \\ [25pt] % Your university, school and/or department name(s)
\horrule{0.5pt} \\[0.4cm] % Thin top horizontal rule
\huge Memoria Práctica 2 \\ % The assignment title
\horrule{2pt} \\[0.5cm] % Thick bottom horizontal rule
}
\usepackage[spanish]{babel}
\author{Amanda Fernández Piedra} % Nombre y apellidos
\usepackage[utf8]{inputenc}
\date{\normalsize\today} % Incluye la fecha actual

%----------------------------------------------------------------------------------------
% DOCUMENTO
%----------------------------------------------------------------------------------------

\begin{document}

\maketitle % Muestra el Título

\newpage %inserta un salto de página

\tableofcontents % para generar el índice de contenidos

\listoffigures

%\listoftables

\newpage

\newpage

%----------------------------------------------------------------------------------------
%	CuestiÂŽon 1
%----------------------------------------------------------------------------------------

\section{Análisis de la eficiencia de los algoritmos implementados en la práctica}
En esta práctica hemos podido aprender lo importante que es la eficiencia de los 
algoritmos a la hora de ejecutar nuestros programas. 
Por un lado, tenemos la primera representación diccionarioV1.hxx la cual realiza 
una inserción desordenada de meteoritos en un diccionario y posteriormente se 
realiza una búsqueda secuencial en el mismo diccionario.
Por otro lado, tenemos la segunda representación diccionarioV2.hxx la cual 
después de realizar la inserción desordenada de meteoritos, los ordena mediante 
el algoritmo de ordenación por burbuja. Posteriormente, se realiza una búsqueda 
de los meteoritos mediante el algoritmo de búsqueda binaria.

\begin{itemize}
  \item diccionarioV1
  \begin{itemize}
    \item Inserción: $45,637$ segundos
    \item Búsqueda en el peor caso de que no encuentre el meteorito en el 
    diccionario: $10$ milisegundos
  \end{itemize}

    \item diccionarioV2
    \begin{itemize}
    \item Inserción: $45,637$ segundos
    \item Ordenación: $87,520$ segundos
    \item Búsqueda en el peor caso de que no encuentre el meteorito en el 
    diccionario: $0,0010$ milisegundos
  \end{itemize}
  \end{itemize}

\section{Conclusiones}



\end{document}