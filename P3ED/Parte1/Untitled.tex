\input{preambuloSimple.tex}

%----------------------------------------------------------------------------------------
%	TÍTULO Y DATOS DEL ALUMNO
%----------------------------------------------------------------------------------------

\title{	
\normalfont \normalsize 
\textsc{{\bf Estructuras de Datos (2014-2015)} \\ Grado en Ingeniería Informática \\ Universidad de Granada} \\ [25pt] % Your university, school and/or department name(s)
\horrule{0.5pt} \\[0.4cm] % Thin top horizontal rule
\huge Memoria Práctica 3 Parte 1\\ % The assignment title
\horrule{2pt} \\[0.5cm] % Thick bottom horizontal rule
}
\usepackage[spanish]{babel}
\author{Amanda Fernández Piedra} % Nombre y apellidos
\usepackage[utf8]{inputenc}
\date{\normalsize\today} % Incluye la fecha actual

%----------------------------------------------------------------------------------------
% DOCUMENTO
%----------------------------------------------------------------------------------------

\begin{document}

\maketitle % Muestra el Título

\newpage %inserta un salto de página

%\tableofcontents % para generar el índice de contenidos

%\listoffigures

%\listoftables

\newpage

\newpage

\section{DiccionarioV1.hxx}

\begin{itemize}

\item \textbf{Diferencias:} En la primera versión del diccionario de meteoritos la única diferencia a destacar es que tanto los métodos insertar como los dos operator[] que realizan consultas al diccionario deben implementar una función de buscar. En la solución que se nos da se implementa la función buscar una vez por cada método, incluso habiendo un método buscar. En mi solución, sólamente he implementado un método buscar y los métodos insertar y operator[] llaman a este método buscar cada vez que lo requieran, de forma que ahorro varias líneas de código.
\item \textbf{Dificultades:} En esta versión no he encontrado ninguna dificultad. Cabe destacar que es la versión más ineficiente y también la más fácil de implementar.

\end{itemize}

\section{DiccionarioV2.hxx}

\begin{itemize}

  \item \textbf{Diferencias:} En la segunda versión sí hay más diferencias. La implementación correcta (según la solución proporcionada) es 
  hacer una inserción ordenada de los meteoritos haciendo uso de el método de 
  búsqueda binaria, lo cual es evidentemente mucho más eficiente que la primera 
  versión del problema. En mi caso, la inserción la he realizado de la misma 
  forma en las dos versiones, pero ésta versión después de insertar, ordena el 
  diccionario con el método de ordenación por burbuja y aunque se gana en 
  eficiencia en el método de búsqueda, se pierde tiempo ordenando el vector 
  pudiendo hacer una inserción ordenada tal y como se proporciona en la 
  solución.
  \item \textbf{Dificultades:} En realidad no he encontrado dificultades, creo 
  que el error que cometí al realizar la práctica fue que no entendí muy bien lo 
  que se pedía en la segunda implementación, mi solución no es tan eficiente.

\end{itemize}

\end{document}